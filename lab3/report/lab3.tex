\documentclass[10pt] {article}
\usepackage{fourier}
\author{Ankit Goyal \\ankit@cs.utexas.edu \\ CS380L}
\title{Lab 3: Program loading and memory mapping}
\date{\today}	
\usepackage{full page}
\usepackage{minted} % to insert code
\renewcommand\listingscaption{Codeblock}



\usepackage{hyperref, url}
\usepackage{listings}
\usepackage{graphicx}
\usepackage{caption}
\usepackage{subcaption}
\usepackage{amsmath}
%\usepackage{amsmath, enumerate, url, ulem, algorithmic, polynom, subfig}

\setlength\parindent{0pt}

\begin{document}
\maketitle
%----------------------------------------------------------------------------------------
%  Specs
%----------------------------------------------------------------------------------------

\section{Setup}
\subsection{Hardware}
\textbf{Host Processor}: 64 bit 4 core Intel(R) Xeon(R) CPU E3-1270 V2 @ 3.50GHz\\
\textbf{Host Memory}: 16GB \\
\textbf{HyperThreading}: Yes \\
\textbf{Logical CPUs after Hyperthreading}: 8 \\
\textbf{CPU frequency scaling}: Disabled in BIOS (turned off Intel SpeedStep and C-states)

\subsection{Software}
\textbf{Host Operating System}: Ubuntu with 3.13.0-34-generic 64 bit kernel.\\

\section{Creating the memory image of a new process}

\texttt{sys\_execve} is responsible for setting up the environment for running the program. Below are the steps taken by \texttt{sys\_execve} which calls  \texttt{do\_execve\_common}:

\begin{enumerate}
\item Check that \texttt{NPROC} limit is not exceeded (i.e., total number of process), if it is then exit. (L: 1443)
\item Allocate memory for data structure in kernel. (L: 1458) 
\item Open the \texttt{exec} file using \texttt{do\_open\_exec} (L: 1469)
\item Now the kernel data structures are initialized and \texttt{exec\_binprm} is called. 
\item \texttt{exec\_binprm} calls \texttt{search\_binary\_handler} which finds the binary format handler, in our case elf. So it finds \texttt{load\_elf\_binary}. (fs/binfmt\_elf.c L:84 \& 571)
\begin{itemize}
\item \texttt{load\_elf\_binary} does consistency checks by making sure that it's an ELF format file by comparing the main number and ELF in \texttt{e\_ident} field in header.
\item \texttt{load\_elf\_binary} reads the header information and looks for \texttt{PT\_INTERP} segment to see if an interpreter was specified. This segment is only present for dynamically linked programs and not for statically linked.
\end{itemize}
\end{enumerate}

\noindent \textbf{Time Spent on the lab \ensuremath{\approx} 30 hours} 

\section{References}
\begin{enumerate}
  \item http://eli.thegreenplace.net/2012/08/13/how-statically-linked-programs-run-on-linux/
  \item http://www.skyfree.org/linux/references/ELF\_Format.pdf
  \item http://linux.die.net/man/5/elf
  \item http://articles.manugarg.com/aboutelfauxiliaryvectors.html
  \item  http://pubs.opengroup.org/onlinepubs/009695399/functions/sigaction.html
  \item http://stackoverflow.com/questions/8116648/why-is-the-elf-entry-point-0x8048000-not-changeable
  \item http://lxr.free-electrons.com/source/fs/exec.c\#L1425
\end{enumerate}

\end{document}